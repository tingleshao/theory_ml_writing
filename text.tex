% ----------------------------------------------------------------
% Article Class (This is a LaTeX2e document)  ********************
% ----------------------------------------------------------------
\documentclass[12pt]{article}
\usepackage[english]{babel}
\usepackage{amsmath,amsthm}
\usepackage{amsfonts}
% THEOREMS -------------------------------------------------------
\newtheorem{thm}{Theorem}[section]
\newtheorem{cor}[thm]{Corollary}
\newtheorem{lem}[thm]{Lemma}
\newtheorem{prop}[thm]{Proposition}
\theoremstyle{definition}
\newtheorem{defn}[thm]{Definition}
\theoremstyle{remark}
\newtheorem{rem}[thm]{Remark}
\usepackage{amssymb}
\numberwithin{equation}{section}
% ----------------------------------------------------------------
\begin{document}

\title{The Derivation of Generalization Bounds for Distance Weighted Discrimination}%
\author{Chong Shao}%
%\address{}%
%\thanks{}%
%\date{}%
% ----------------------------------------------------------------

\maketitle
% ----------------------------------------------------------------
\section{Introduction}
Probably Approximate Correct (PAC) learning framework is a useful tool in examining the performance of a learning algorithm with respect to the sample, time and space complexity. It builds the theoretical foundation of the area of machine learning. Tools like Radmacher complexity and VC-dimension are used in order to present the generalization bound of a learning algorithm in some cases. In previous work, the generalization bound for Support Vector Machines (SVM) are successfully derived. In this paper, the author is trying to derive a generalization bound for a different binary classification learning algorithm: Distance Weighted Discrimination (DWD). The second section sums up the technologies used in proving the generalization bound. The third section gives a example of deriving generalization bound for SVM. The fourth one gives another derivation for DWD.
\section{PAC Learning Model}
Probably Approximate Correct (PAC) learning framework was introduced by Valiant [1], to help define the class of learnable concepts in terms of sample complexity, time complexity and space complexity of a learning algorithm [2]. In the previous definition of PAC-learning, sample complexity is related to the number of training sample points needed to achieve an approximate solution.
\subsection{Definitions}
Several definitions need to be introduced before introducing the PAC learning model. \\[0.2cm]
1. $\mathcal{X}$: the set of all possible examples.\\[0.2cm]
2. $\mathcal{Y} = \{0,1\}$: set of all possible labels in binary classification. \\[0.2cm]
3. A \emph{concept} \(c: \mathcal{X} \rightarrow \mathcal{Y}\): a mapping from \(\mathcal{X}\) to \(\mathcal{Y}\). \\[0.2cm]
4. A \emph{concept class} $C$: a set of concepts.  \\[0.2cm]
5. Assume that examples are independently and identically distributed (i.i.d) according to some fixed distribution $D$. $D$ is unknown. \\[0.2cm]
6. The learning algorithm receives a set of training examples $S = (x_1, \dots , x_m)$ and the corresponding labels $C=(c(x_1), \dots, c(x_m))$. $c \in C$ is called the \emph{target concept} that we want to learn. \\[0.2cm]
7. The learning algorithm will use the training examples to produce a hypothesis $h_S \in H$ selected from a hypothesis set $H$ that maps $\mathcal{X}$ to $\mathcal{Y}$. Note that $H$ and $C$ may not be the same.\\[0.2cm]
8. It is considered good if yielded hypothesis $h_S$ has a small generalization error. \\[0.2cm]
9. Definition of \textbf{Generalization error}:[2] \\[0.2cm]
Given a hypothesis $h \in H$, a target concept $c \in C$, and an underlying distribution $D$, the generalization error $R(h)$ is defined by
\[R(h)= \underset{x\sim D}{\text{Pr}}[h(x)\neq c(x)]= \underset{x \sim D}{\text{E}}[1_{h(x) \neq c(x)}]\]
The generalization error $R(h)$ cannot be directly computed since the concept $c$ is unknown. However the \emph{empirical error} of a hypothesis can be computed to estimate the generalization error. \\[0.2cm]
10. Definition of \textbf{Empirical error}: [2] \\[0.2cm]
Given a hypothesis $h \in H$, a target concept $c \in C$, and a sample $S = (x_1,\dots,x_m)$ with corresponding labels $C = (c(x_1),\dots,c(x_m))$, the generalization error $R(h)$ is defined by
\[\widehat{R}(h)= \frac{1}{m}\sum_{i=1}^{m}1_{h(x_i) \neq c(x_i)}\]
It can be proved that the expectation of empirical error based on an i.i.d. sample $S$ equals to the generalization error.
\begin{align*}
 \underset{S \sim D^m}{\text{E}}[\widehat{R}(h)] &= \frac{1}{m}\sum_{i=1}^{m}\underset{S \sim D^m}{\text{E}}[1_{h(x_i) \neq c(x_i)}] \\
 &= \frac{1}{m}\sum_{i=1}^{m}\underset{S \sim D^m}{\text{E}}[1_{h(x) \neq c(x)}] \text{  (Because samples are drown i.i.d.)} \\
 &= \underset{S \sim D^m}{\text{E}}[1_{h(x) \neq c(x)}] \\
 &= \underset{x \sim D}{\text{E}}[1_{h(x) \neq c(x)}] \\
 &= R(h)
\end{align*}
\subsection{PAC Model}
[2] A concept class $C$ is called PAC-learnable if the hypothesis returned by the algorithm after observing a number of points in polynomial of $1/\epsilon$ and $1/\delta$ has error at most $\epsilon$ with probability at least $1-\delta$. Since $\epsilon$ and $\delta$ are small, it is approximately correct with high probability. \\[0.2cm]
\textbf{Formal definition}:\\[0.2cm]
A concept class $C$ is said to be PAC-learnable if there exists an algorithm $\mathcal{A}$ and a polynomial function such that for any $\epsilon > 0$ and $\delta > 0$, for all distributions $D$ on $\mathcal{X}$ and for any target concept $c \in C$, for following inequality holds for any sample size $ m \geq poly(1/\epsilon,1/\delta,n,size(c) )$:
\[\underset{S \sim D^m}{\text{Pr}}[R(h_S) \leq \epsilon ] \geq 1 - \delta \]
Note that PAC does not make assumption on the distribution D. For an algorithm, the concept class $C$ is determined but a particular concept $c$ is unknown. \\[0.2cm]
(It can be shown that this inequality can be used to bound generalization error)
\subsection{Example (later)}
1. finite H, consistent case \\[0.2cm]
can get a inequality for $R(h)$\\[0.2cm]
2. finite H, inconsistent case \\[0.2cm]
can get a inequality for $R(h)$ \\[0.2cm]
Link to the idea of Occam's razor principle (later)
\subsection{Rademacher Complexity}
Rademacher Complexity: \\[0.2cm]
In order to produce generalization bound for infinite hypothesis sets, (for example, lines in 2D plane), some tools are required. One is Rademacher Complexity. \\[0.2cm]
Definition of \textbf{Empirical Rademacher complexity}: \\[0.2cm]
Let $G$ be a family of functions mapping from $Z$ to $[a,b]$ and $S=(z_1,\dots,z_m)$ a fixed sample of size $m$ with elements in $Z$. And we write $\mathbf{g}_S = (g(z_1),\dots,g(z_m))^T$. The empirical Rademacher complexity of $G$ with respect to $S$ is defined as:
\[\mathfrak{\widehat{R}}_S(G)=\underset{\mathbf{\sigma}}{\text{E}}[\underset{g \in G}{\text{sup}}\frac{1}{m}\sum_{i=1}^{m}\sigma_ig(z_i)]=\underset{\mathbf{\sigma}}{\text{E}}[\underset{g \in G}{\text{sup}}\frac{\mathbf{\sigma}\cdot\mathbf{g}_S}{m}]\]
$\mathbf{\sigma} = (\sigma_1,\dots, \sigma_m)^T$. And $\sigma_i$'s are independent uniform random variables taking values $\{-1,+1\}$. Like tossing a fair coin m times. \\[0.2cm]
The definition can be understood as a measure of how well the function class $G$ correlates with $\mathbf{\sigma}$ over the sample $S$. Since $\mathbf{\sigma}$ can be understood as noise, Empirical Rademacher complexity measures the richness of function class $G$. A richer or more complex function class $G$ can generate more $\mathbf{g}_S$ and will on average better correlate with random noise.\\[0.2cm]
Definition of \textbf{Rademacher Complexity}: \\[0.2cm]
For any integer $m\geq 1$, let $G$ be a family of functions mapping from $Z$ to $[a,b]$ and $S=(z_1,\dots,z_m)$ a fixed sample of size $m$ with elements in $Z$. The Rademacher complexity of $G$ is defined as:
\[\mathfrak{R}_m(G)=\underset{S \sim D^m}{\text{E}}[\mathfrak{\widehat{R}}_S(G)]\]
\textbf{Generalization bound for binary classification using Rademacher complexity}:\\[0.2cm]
Let $H$ be a family of functions taking values in $\{-1,+1\}$. Let $D$ be the distribution over the input space $\mathcal{X}$. For any $\delta > 0$, with probability at least $1-\delta$ over a sample $S$ of size $m$ drawn according to $D$, each of the following holds for any $h \in H$:
\[R(h)\leq \widehat{R}(h) + \mathfrak{R}_m(H) + \sqrt{\frac{\mathrm{log}\frac{1}{\delta}}{2m}}\]
\[R(h)\leq \widehat{R}(h) + \mathfrak{\widehat{R}}_S(H) + 3\sqrt{\frac{\mathrm{log}\frac{2}{\delta}}{2m}}\]
Proof: (later)
\subsection{VC-Dimension}
In some cases, computing $\mathfrak{\widehat{R}}_S(H)$ is computationally hard. VC-Dimension is introduced as an alternative. \\[0.2cm]
\textbf{Definition}
The VC-dimension of a hypothesis set $H$ is the size of the largest set that can be fully shattered by $H$:
\[VCdim(H) = max\{m:\Pi_H(m) = 2^m\}\]
\textbf{Example: Hyperplanes}
\textbf{Generalization bound for binary classification using VC-Dimension}
Let $H$ be a family of functions taking values in $\{-1,+1\}$ with VC-dimension d. Then for any $\delta > 0$, with probability at least $1-\delta$, the following holds for all $h \in H$:
\[R(h) \leq \widehat{R}(h) + \sqrt{\frac{2d\mathrm{log}\frac{em}{d}}{m}} + \sqrt{\frac{\mathrm{log}\frac{1}{\delta}}{2m}}\]
Proof: (later) 
\section{Support Vector Machines}
Support Vector Machines (SVM) is a widely used tool for binary classification, invented by Vapnik. The SVM algorithm for general non-separable case was introduced by Cortes and Vapnik [3].  
\subsection{Formulation}
For SVM algorithm, the hypothesis set is the set of hyperplanes.
\[H = \{x\mapsto \mathrm{sign}(\mathbf{w\cdot x}+b): \mathbf{w} \in \mathbb{R}^N, b \in \mathbb{R} \}\] 
For a linearly separable data set, the goal is to find a hyperplane that successfully separate two group of points with labels $\{-1,+1\}$that, points with the same label will be on the same side of hte separating hyperplane. Here is the corresponding primal optimization problem: 
\[\underset{\mathbf{w}, b}{\mathrm{min}} \frac{1}{2}\|\mathbf{w}\|^2\]
\[\text{subject to: } y_i(\mathbf{w\cdot x_i}+b) \geq 1, \forall i \in [1,m]\] 
Note that in the problem we can scale $\mathbf{w}$ and $b$ to make $\underset{(x,y)\in S}{\mathrm{min}}|\mathbf{w\cdot x}+b| =1 $.
For this problem, we can write the Lagrangian:
\[\mathcal{L}(\mathbf{w},b,\mathbf{\alpha})=\frac{1}{2}\|\mathbf{w}\|^2-\sum_{i=1}^{m}\alpha_i [y_i(\mathbf{w\cdot x_i}+b)-1]\]
\subsection{Margin Theory}
bla
\section{Distance Weighted Discrimination}
\subsection{Formulation}
\subsection{Weighted DWD}
www

\section{References}
[1] Vapnik 1984 \\[0.2cm]
[2] Mohri et.al. 2012 \\ [0.2cm]
[3] Cortes, Corinna; and Vapnik, Vladimir N.; "Support-Vector Networks", Machine Learning, 20, 1995.\\ http://www.springerlink.com/content/k238jx04hm87j80g/
\end{document}
% ----------------------------------------------------------------
